\documentclass{article}
\usepackage[utf8]{inputenc}

\title{Résumé de : "A Deep Reinforcement Learning Framework for the
Financial Portfolio Management Problem"}
\author{Michaël TRAZZI}

\usepackage{natbib}
\usepackage{graphicx}

\begin{document}

\maketitle

\section{Introduction}
Il y a quatre théories fondamentales pour la gestion du portfolio: "Follow the winner", "Follow the loser", "Pattern-Matching" et "Meta-learning". Il existe des approches Deep Learning appliquées au marchés financier, mais celles-ci se focalisent sur la prédiction du prix, ce qui s'avère particulièrement difficile pour le cas des cryptomonnaies. Les travaux d'apprentissage concernant l'algorithmic trading n'essayant pas de prédire le prix, et n'utilisant pas de modèle pré-existant, utilisent des techniques de Reinforcement Learning(RL). Cependant, ces travaux se contentent d'échanger seulement un stock, ce qui ne nous intéresse pas pour le cas du porfolio. Les techniques de Deep RL récentes (David Silver, 2016) consistant à entraîner deux politiques en même temps peuvent s'avérer trop difficiles et instables. C'est pourquoi dans ce papier est adoptée une approche de RL adaptée au cas du portfolio. Cette approche utilise des réseaux de neurones (RNN, CNN et LSTM) prenant en entrée l'historique de stocks, pour prédire l'action adaptée à adopter à la fin de la période (e.g. acheter, vendre, etc.). Les résultats théoriques sont testés en pratique en effectuant un trading permanent sur Polonix.com, à intervalles de 30 minutes. Contrairement au marché habituel, le marché des cryptomonnaies est ouvert 24h/24 7j/7.

\section{Définition du problème}



\section{Conclusion}


\end{document}
